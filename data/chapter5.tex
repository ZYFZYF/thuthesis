% !TeX root = ../main.tex

\chapter{总结与展望}
\label{cha:intro}
本文实现了基于深度学习的时间序列异常检测框架和基于时间序列异常检测的根因分析系统两个部分的工作,前者主要是在一个合理的评价标准以及统一的数据集上实现了已有的多种算法,也对已有的单点异常检测算法与时序模型结合提出了一些新的模型,在效果上有小幅的提升,但随之带来的是耗时的大幅增长,总体来讲并不划算,但该框架可用于之后与其他算法的横向比较;后者主要不同于以往的工作以时间序列数据的相关性为边权,我们提取出和边有关的时间序列数据(例如调用次数、返回耗时等),将时间序列异常检测的结果作为边权,结合点上的时间序列异常检测提供的点权,在有网络拓扑的前提下构建出了异常传播图,在图上运行随机游走算法来找根因,在AIOPS2020挑战赛中进行验证,得到了0.91的准确率。

同时,本文还有很多值得深入研究的地方。未来进一步的研究工作为:
\begin{enumerate}
    \item 没有将前后两者的工作形成一个完整的整体。前者的时间序列异常检测模型应该服务于后者的根因分析系统,但在评测时由于数据量的问题我们在根因分析系统中没能采用基于深度学习的方法;
    \item 时间序列异常检测的框架只采用了SMD一个数据集,虽然是出于其与云网络的场景较为相似而使用,但算法的效果不应该依赖于数据集的类型,只用一个数据集得出的结果很有可能有偶然性,可以增加数据集的使用,研究算法在不同数据集下的通用性;
    \item POT的方法在实验中发现对超参的选取十分敏感,同时耗时也不稳定,所以在落地上仍然具有一定的困难,还需继续探索其他自动确立阈值的方法;
    \item 将设计的系统实际部署到云网络中进行使用,还要考虑数据量的问题、通用性、检测效率等等问题是否满足要求。
\end{enumerate}

