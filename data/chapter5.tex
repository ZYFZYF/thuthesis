% !TeX root = ../main.tex

\chapter{总结与展望}
\label{cha:summary}
本文实现了基于深度学习的时间序列异常检测框架和基于时间序列异常检测的根因分析系统两个部分的工作,前者主要是在一个合理的评价标准以及统一的数据集上实现了已有的多种算法,也对已有的单点异常检测算法与时序模型结合提出了一些新的模型,在效果上有小幅的提升,但随之带来的是耗时的大幅增长,总体来讲并不划算,但该框架可用于之后与其他算法的横向比较;后者主要不同于以往的工作以时间序列数据的相关性为边权,本文提取出和边有关的时间序列数据,直接将时间序列异常检测的结果作为边权,结合点上的时间序列异常检测提供的点权,增加反向边和自环,建出了异常传播图,在图上运行随机游走算法来找根因,在公开数据集中进行验证,得到了较高的准确率。

同时,本文还有很多值得深入研究的地方。未来进一步的研究工作为:
\begin{enumerate}
    \item 时间序列异常检测的框架采用了SMD一个数据集,可以增加数据集的使用,研究算法在不同数据集下的通用性;
    \item 在实验中发现POT方法对超参的选取十分敏感,同时耗时也不稳定,所以在落地上仍然具有一定的困难,还需继续探索其他自动确立阈值的方法;
    \item 将设计的两个系统结合起来,使用深度学习的方法进行异常检测,结合异常检测的结果进行根因分析,将得到的系统实际部署到云网络中进行使用,考察数据规模、通用性、检测效率、检测效果等方面是否满足要求。
\end{enumerate}
