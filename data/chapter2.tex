% !TeX root = ../main.tex

\chapter{相关研究综述}
\label{cha:intro}
\section{时间序列异常检测技术}
时间序列异常检测是一个很经典的问题。传统方法主要基于统计,最常用也最高效的是基于3$\sigma$原则,基于历史的数据分布来确定当前数据的合理波动范围,它假设数据分布满足正态分布,因此不在$(\mu - 3\sigma,\mu + 3\sigma)$范围内的数据极有可能是异常值。但实际数据可能不符合正态分布的假设,具体使用的标准差倍数也很难统一。另一种常用的是ARIMA\footnote{Autoregressive Integrated Moving Average model,差分整合移动平均自回归模型}方法,该方法适用于平稳、少突降/突增的数据,通过前一段的时间的数据来预测下一个时刻的数据,然后通过比较预测值和真实值的差异来判断异常的发生,但该方法有7个参数需要确定,不同的KPI难以找到一套自动化的流程来确定参数。

近年来,随着人工智能技术的发展,越来越多的深度学习技术被用到了这一领域。而深度学习按是否有标数据又分为有监督学习和无监督学习。鉴于时间序列异常检测的特殊性:标签难以获得以及难以穷尽所有的异常。所以目前无监督方法更合适,也就是用全是正常的数据训练而在测试时数据中会夹杂异常。Malhotra\cite{malhotra2015long}首次将LSTM用于时间序列异常检测这一领域,并且假设一段时间内的预测误差符合高斯分布;An\cite{an2015variational}利用VAE的特点对正常数据用重构的方法来做异常检测,并且提出了重构概率的概念;Xu\cite{xu2018unsupervised}则进一步将VAE用到了时间序列异常检测上;Ruff\cite{ruff2018deep}则用神经网络来实现OC-SVM来解决传统OC-SVM在面临高维数据时表现差的问题;Malhotra\cite{malhotra2016lstm}则将预测与重构结合起来,提出了LSTM-Based Encoder Decoder,解决了在面临某些无法预测的时间序列的时候的问题;Zong\cite{zong2018deep}则将AutoEncoder与传统的GMM模型结合提出了一个端到端的模型;Park\cite{park2018multimodal}则将LSTM和VAE结合起来,规避了VAE本身并不是一个时序模型的特点;Su\cite{su2019robust}则进一步考虑了VAE中随机变量的时序性的特点,目前实现了state-of-the-art的结果。

\section{复杂系统中的根因定位}

在云网络上的多点异常分析上,Lin\cite{lin2016automated}等人提出了异常传播路径的概念,将多点之间的关联边分为水平边和竖直边,并构建出异常传播图,在此基础上到所有异常点的距离之和就作为评判一个点是根因的依据。Weng\cite{weng2018root}等人则在此基础上做出了改进,并主要针对的是公共云中的多级服务调用场景,在图中对边的类型进行了区分并且引入了边权,在计算时也用随机游走来更真实地模拟真实的异常传播,实现了更加细致和具体场景下的根因分析。\cite{wu2020microrca}则进一步使用了Personal PageRank来更加高效和准确的计算根因。
\section{小结}
本章对时间序列异常检测算法以及复杂系统中的根因定位进行了综述。

