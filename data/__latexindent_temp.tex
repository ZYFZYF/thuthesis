% !TeX root = ../main.tex

% 中英文摘要和关键字

\begin{abstract}
  近些年来越来越多的企业和个人选择将自己的服务部署在云网络中,云网络的规模随之扩大,而且伴随而来的还有经常发生的大规模故障。当这些故障在复杂咋云网络场景中发生时,及时地检测和定位就成了一个难题。

  本文以云网络为背景抽象出了两个问题进行研究,所提出的两个系统结合起来理论上可以在云网络中进行高效而准确的异常检测和根因定位。其一是用深度学习的方法对多元时间序列数据进行异常检测的框架,可以部署在云网络中的单点上,对每个时刻的多条指标检测曲线进行异常的发现,本文实现了多种算法并在ServerMachineDataset\cite{su2019robust}数据集上进行了测试和比较;其二是在复杂系统上的根因分析系统,可以部署在某一个闭环系统上,结合单点的异常情况和节点之间的拓扑关系找出异常发生的根本源头,本文提出的方法基于异常检测和随机游走,目前在AIOPS2020挑战赛的前两阶段中可以获得100\%的正确率。

  % 关键词用“英文逗号”分隔
  \thusetup{
    keywords = {云网络,异常检测,根因分析,随机游走,深度学习},
  }
\end{abstract}

\begin{abstract*}
  In recent years, more and more enterprises and individuals have chosen to deploy their services in the cloud network, and the scale of the cloud network has expanded accordingly, and it is accompanied by frequent large-scale failures. When these faults occur in a complex cloud network scenario, timely detection and location becomes a problem.

  In this paper, two problems are abstracted and studied based on the cloud network. The two systems proposed can theoretically perform efficient and accurate anomaly detection and root cause location in the cloud network. One is a framework for anomaly detection of multivariate time series data using deep learning methods, which can be deployed at a single point in the cloud network to discover anomalies for multiple index detection curves at each moment. The algorithm is tested and compared on the ServerMachineDataset dataset; the second is the root cause analysis system on complex systems, which can be deployed on a closed-loop system, combining single-point anomalies and the topological relationship between nodes to find the root cause of anomalies. The method proposed in this paper is based on anomaly detection and random walk. At present, 100\% accuracy rate can be obtained in the first two stages of the AIOPS2020 challenge.
  \thusetup{
    keywords* = {cloud network,anomaly detection, root cause analysis, random walk, deep learning},
  }
\end{abstract*}
