% !TeX root = ../main.tex

\chapter{相关研究综述}
\label{cha:intro}
\section{时间序列异常检测技术}
时间序列异常检测是一个很经典的问题。传统方法主要基于统计,最常用也最高效的是基于3$\sigma$原则,基于历史的数据分布来确定当前数据的合理波动范围,它假设数据分布满足正态分布,因此超出$(\mu + 3\sigma,\mu-3\sigma)$范围的数据极有可能是异常值。但实际数据可能不符合正态分布的假设,具体使用的标准差倍数也很难统一。另一种常用的是ARIMA\footnote{Autoregressive Integrated Moving Average model,差分整合移动平均自回归模型}方法,该方法适用于平稳、少突降/突增的数据,通过前一段的时间的数据来预测下一个时刻的数据,然后通过比较预测值和真实值的差异来判断异常的发生,但该方法有7个参数需要确定,不同的KPI难以找到一套自动化的流程来确定参数。

近年来,随着人工智能技术的发展,越来越多的深度学习技术被用到了这一领域。而深度学习按是否有标数据又分为有监督学习和无监督学习。鉴于时间序列异常检测的特殊性:标签难以获得以及难以穷尽所有的异常。所以目前无监督方法更合适。而用无监督方法做时间序列异常检测的方法又主要有两种思路:基于预测和基于重构的。

基于预测的方法

\section{复杂系统中的根因定位}
\section{小结}

