% !TeX root = ../main.tex

\chapter{引言}
\label{cha:intro}

\section{研究背景}
近几年,由于互联网业务的快速发展,越来越多的企业和个人选择将自己的服务部署在云平台上。云网络最大的特点就是虚拟化,通过互联网来提供动态易扩展且经常是虚拟化的资源,租户以为自己在独占物理资源,而时常它们在与其他租户共享一台物理设备。这样确实更加方便扩展,充分利用资源,但随之而来的是网络以及节点状态的复杂化。不仅虚拟机之间因为调用关系会互相影响,因为落在同一台物理机上,虚拟机之间也会因为资源的争夺而产生影响。

当云网络中发生异常时,其带来的影响是巨大的。因为云网络承载了诸多企业和个人的云计算服务,当网络发生异常时,必然会影响到这些企业产品的正常运行,如果不及时检测/解决的话还会造成更严重的异常扩散,会给云网络提供商带来巨大的经济损失、舆论压力以及可信度的下降。

因此,在云网络中进行异常检测是必要的。及时检测并通报异常能够让我们尽快地防止异常扩散,减少经济损失;如果能发现潜在的异常更是能让我们尽早采取措施,合理分配资源,将可能发生的威胁扼杀在摇篮之中。另一方面,当产生大规模异常报警时,如果我们的异常检测系统能够辅助我们进行根因分析,更能减少对于异常的排查压力,帮助系统快速恢复。
\section{主要工作}
本文的研究工作分为两个部分:

\begin{enumerate}
    \item 提出了一个基于深度学习方法的时间序列异常检测框架,内含多数算法实现、完整的运行流程和一套高效、合理的评价体系;
    \item 将时间序列异常检测与随机游走结合起来,提出了一个根因分析的系统设计与实现。在AIOPS2020中投入尝试,目前可以准确定位到所有根因。
\end{enumerate}
\section{论文结构}
本文共有五章:
\begin{itemize}
    \item 第一章介绍研究背景和主要工作;
    \item 第二章介绍相关工作;
    \item 第三章介绍基于深度学习的时间序列异常检测框架设计与实现,并且对各部分的算法选择进行了详细说明;
    \item 第四章介绍基于时间序列异常检测的根因分析系统设计与实现;
    \item 第五章总结本文并揭示目前工作存在的一些不足。
\end{itemize}