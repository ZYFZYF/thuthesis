% !TeX root = ../main.tex

\chapter{引言}
\label{cha:intro}

\section{研究背景}
近几年,由于互联网业务的快速发展,越来越多的企业和个人选择将自己的服务部署在云网络上,网络规模随之增加的同时也带来了难以分析的大规模网络故障。因此在云网络中进行异常检测和根因定位对维持云网络稳定运行有着重大意义。

云网络的异常检测需要运维人员监控云网络中一个节点的状态。在每个时刻,运维人员会得到该点的一些指标,例如CPU利用率、内存利用率、网络吞吐量等,运维人员需要结合过往的指标值来推测这一时刻该节点的各项指标是否异常。这一问题可以被抽象为一个多元时间序列数据的异常检测问题。传统的基于统计的方法存在通用性不够强、准确率较低的问题;而工业界上目前用的最多的为各个指标人为设定一个根据经验得来的阈值也存在耗费人力以及大量误报的问题。近些年不少学者\cite{an2015variational,malhotra2015long,malhotra2016lstm,nguyen2018anomaly,park2018multimodal,ruff2018deep,su2019robust,zong2018deep,xu2018unsupervised,siffer2017anomaly}将深度学习的方法用到了这一领域。深度学习具有流程自动化、不需要人为参与的同时又能达到较高的准确率的优点。尽管目前这方面的工作很多,但缺乏一个统一的运行环境和评价标准。学者们通常是各自用自己的数据集和评判标准,难以进行有说服力的横向比较,因此本文设计了一个通用的基于深度学习方法的多元时间序列数据异常检测框架,并提供了高效、合理的评价方式来进行算法之间的综合比较。

根因分析的任务则是需要运维人员在复杂系统中产生多点有关联的故障时,找出故障发生的根因。由于系统中的调用关系错综复杂,单点的故障很容易扩散出现大规模组件的异常,使得难以整体分析和快速进行故障恢复。近几年的相关工作\cite{lin2016automated,weng2018root,wu2020microrca},基本思想都是直接或间接获取到节点之间的调用关系,通过用时间序列数据之间的相关性来构造出一张异常传播图,最后再用随机游走之类的方法来模拟异常传播或者追溯根因的方向来找到根因。本文在构造传播图时充分结合了异常检测的结果,使得结果解释性更强,也更为准确。


\section{主要工作}
本文的研究工作主要如下:

\begin{enumerate}
    \item 用深度学习的方法对单点的多维时间序列数据进行异常检测。具体包括:
        \begin{itemize}
        \item 设计了一个适用于深度学习方法的多元时间序列异常检测框架,使用统一的数据集和评价方式进行评估,对各个算法进行了性能和效率方面的综合比较;
        \item 选择并实现了LSTM\footnote{Long Short-Term Memory,长短期记忆人工神经网络}、AE\footnote{AutoEncoder,自编码器}、VAE\footnote{Varitional Autoencoder,变分自编码器}\cite{an2015variational}、DAGMM\footnote{Deep Autoencoding Gaussian Mixture Model,深度自编码的高斯混合模型}\cite{zong2018deep}、Deep SVDD\footnote{Deep Support Vector Data Description,深度化的支持向量数据描述}\cite{ruff2018deep}等算法;
        \item 将单点异常检测算法与深度学习的时间序列模型结合改进提出了LSTM-VAE、LSTM-AE、LSTM-DAGMM以及设计实现了将重构和预测结合起来的AE-Predictor算法,并进行了与原模型实验结果的对比;
        \item 采用了由极值理论推导得出的POT\cite{siffer2017anomaly}方法来自动确立算法的阈值,而不依赖于验证集以及对原数据的强分布假设;
        \item 实现了针对时间序列数据的Precision\footnote{精确率}和Recall\footnote{召回率}的新型计算方式\cite{tatbul2018precision},并且在枚举所有阈值计算Precision和Recall的时候用并查集和增量更新的方式使得计算复杂度下降了一个数量级。
        \end{itemize}
    \item 用随机游走的方式结合异常检测的结果进行根因的定位,具体包括:
        \begin{itemize}
        \item 提出了一个将时间序列异常检测的结果直接应用于异常传播图的构建中、与随机游走相结合的方法,可解释性更强;
        \item 在公开数据集上进行实验,对算法的准确率进行了评估,并展示了算法的可解释性。
        \end{itemize}
\end{enumerate}
\section{论文结构}
本文内容分为五章:
\begin{itemize}
    \item 第一章介绍本文的研究背景和主要工作;
    \item 第二章介绍相关工作,主要是异常检测和根因分析两方面,前者主要介绍近些年用深度学习做异常检测的工作,后者则介绍通过构建调用图和随机游走的方法进行根因分析的工作;
    \item 第三章介绍本文设计实现的基于深度学习的时间序列异常检测框架,并且对框架的各部分进行详细的介绍;
    \item 第四章介绍本文设计实现的基于时间序列异常检测的根因分析系统,以AIOPS2020挑战赛为评测数据集进行了结果的展示;
    \item 第五章总结本文,并对未来工作进行展望。
\end{itemize}