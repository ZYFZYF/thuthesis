% !TeX root = ../main.tex

\chapter{引言}
\label{cha:intro}

\section{研究背景}
近几年,由于互联网业务的快速发展,越来越多的企业和个人选择将自己的服务部署在云网络上。随之而来的是网络规模的增加以及频繁发生的大规模网络故障。因此在云网络中进行异常检测和根因定位对维持云网络稳定运行有着重大意义。

异常检测方面,通常我们需要监控云网络中一个节点的状态,在每个时刻,我们可以得到该点的一些指标,例如CPU利用率、内存利用率、网络吞吐量等,我们需要结合过往的指标值来推测这一时刻该节点的表现是否正常,可以将该问题抽象为一个多元时间序列数据的异常检测问题。过去已经有一些方法来解决该问题,但目前工业界上用的最多的还是人为地为各个指标设定一个根据经验得来的阈值,存在耗费人力以及大量误报的问题。近些年不少学者\cite{an2015variational,malhotra2015long,malhotra2016lstm,nguyen2018anomaly,park2018multimodal,ruff2018deep,su2019robust,zong2018deep,xu2018unsupervised,siffer2017anomaly}提出了用深度学习的方法来进行该问题的检测,将流程自动化、不需要人为参与的同时又能达到较高的准确率。但目前的工作缺乏一个统一的运行环境和评价标准,通常是各自用自己的数据集和评判标准,因此本文提出了一个通用的深度学习方法的多元时间序列数据异常检测框架,并提供了多种高效、合理的评价方式来进行算法之间的综合比较。

根因分析的任务则是要求我们在复杂系统中当有多点产生故障时,这些故障通常是相关联的,而且往往是由一个地方的故障扩散到整个系统,找出故障发生的根因。近几年陆续有这方面的工作\cite{lin2016automated,weng2018root,wu2020microrca},基本思想都是通过直接或间接获取到节点之间的调用关系,通过用时间序列数据之间的相关性来构造出一张异常传播图,最后再用随机游走之类的方法来模拟异常传播或者追溯根因的方向,来找到根因。本文在构造传播图时充分结合了异常检测的结果,使得结果解释性更强,也更为准确。


\section{主要工作}
本文的研究工作分为两个部分:

\begin{enumerate}
    \item 针对云网络中的单点异常,本文提出了一个基于深度学习方法的多元时间序列异常检测框架,内部实现了主流的用于异常检测的多种深度学习方法,结合\cite{tatbul2018precision}和\cite{siffer2017anomaly}进行多种方式的评估,并且对计算效率进行了优化;
    \item 针对云网络中的根因定位,本文将时间序列异常检测与随机游走结合起来,提出了一个解释性更强的算法,在AIOPS2020中投入使用,在前两阶段评测数据中获得了100\%的正确率。
\end{enumerate}
\section{论文结构}
本文内容分为五章:
\begin{itemize}
    \item 第一章介绍本文的研究背景和主要工作;
    \item 第二章介绍相关工作,主要是异常检测和根因分析两方面,前者集中在近些年来用深度学习做异常检测的工作,后者集中介绍通过构建调用图和随机游走的方法进行根因分析的工作;
    \item 第三章介绍本文设计实现的基于深度学习的时间序列异常检测框架,并且对框架的各部分进行详细说明;
    \item 第四章介绍本文设计实现的基于时间序列异常检测的根因分析算法,以AIOPS2020挑战赛为背景介绍算法的实现细节;
    \item 第五章总结本文,揭示目前工作存在的一些不足,并对未来工作进行展望。
\end{itemize}