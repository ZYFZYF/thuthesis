% !TeX root = ../main.tex

% 中英文摘要和关键字

\begin{abstract}
  近些年来越来越多的企业和个人选择将自己的服务部署在云网络中,云网络的规模和复杂性逐步增加的同时,也带来了难以分析的大规模故障,这时如何及时地检测故障和准确定位故障根因就成了一个难题。

  本文以云网络为背景抽象出了两个问题进行研究,所提出的两个系统结合起来可以在云网络中进行高效而准确的异常检测和根因定位。第一个是用深度学习的方法对多元时间序列数据进行异常检测的框架,可以部署在云网络中的单点上,对每个时刻的多条指标曲线进行异常的发现,本文实现了多种已有的深度学习算法和一些改进的尝试并在相关数据集上进行了综合测试和比较;第二个则是在云网络上的异常根因分析系统,需要结合单点的异常情况和节点之间的拓扑关系找出异常发生的根本源头,本文提出了一个基于异常检测和随机游走的方法来解决该问题,并在公开数据集上测试获得了较高的准确率。

  % 关键词用“英文逗号”分隔
  \thusetup{
    keywords = {云网络,异常检测,根因分析,随机游走,深度学习},
  }
\end{abstract}

\begin{abstract*}
  In recent years, more and more enterprises and individuals have chosen to deploy their services in the cloud network. As the scale and complexity of the cloud network gradually increase, it also brings large-scale failures that are difficult to analyze. Detecting faults and accurately locating the root cause of the faults have become an urgent problem to be solved.

  In this thesis, two problems are abstracted and studied based on the cloud network. The two systems proposed can theoretically perform efficient and accurate anomaly detection and root cause location in the cloud network. One is a framework for anomaly detection of multivariate time series data using deep learning methods, which can be deployed at a single point in the cloud network to discover anomalies for multiple KPIs(Key Performance Indicator) at each time. This thesis implements a variety of existing deep learning algorithms and some improved attempts and conducts comprehensive tests and comparisons on related data sets; Another is the root cause analysis system on cloud network, which needs to combine single-point anomalies and the topological relationship between nodes to find the root cause of abnormal occurrences. This thesis proposes a method based on anomaly detection and random walk to solve this problem, and the experimental results show a high accuracy rate in a public dataset.

  \thusetup{
    keywords* = {cloud network,anomaly detection, root cause analysis, random walk, deep learning},
  }
\end{abstract*}
